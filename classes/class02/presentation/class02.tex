\documentclass{beamer}

\usetheme{Warsaw}

\usepackage{amsthm}
\usepackage{verbatim}
\usepackage{xcolor}
\usepackage{listings}
\usepackage{color}
\usepackage[utf8]{inputenc}
\usepackage[T1]{fontenc}
\usepackage{textcomp}
\usepackage{graphicx}
\usepackage[bulgarian]{babel}

\usepackage{hyperref}

\title{Класове и пакети, наследяване, абстрактни класове и интерфейси, полиморфизъм, капсулация на данни}
\author{Валентин Гелински}
\institute{ТУ София}
\date{\today} 


\begin{document}


  \begin{frame} 
    \titlepage
  \end{frame}

  \section{Класове}

  \begin{frame} 
    \frametitle{Класове и обекти - отново}
     \begin{block}{Според Wikipedia}
       Класът е съвкупност от променливи и функции, които са обвързани в логическа структура и работят заедно. Класът служи като модел за представяне на реални обекти и софтуерни обекти, описвайки атрибути (свойства) и методи (поведение) на обектите.
     \end{block}
  \end{frame}

  \begin{frame}
    \frametitle{Важни думички}
    \begin{itemize}
      \item{Конструктор}
      \item{this}
      \item{null}
    \end{itemize}
  \end{frame}

  \section{Наследяване}

  \begin{frame}
    \frametitle{Наследяване}
    \begin{block}{Според Wikipedia}
      Inheritance is when an object or class is based on another object or class, using the same implementation (inheriting from a class) or specifying implementation to maintain the same behavior 
    \end{block}
  \end{frame}

  \section{Абстрактни класове и интерфейси}

  \begin{frame}
    \frametitle{Абстрактни класове и интерфейси}
    \begin{itemize}
      \item{Абстрактен клас - клас с поне един абстрактен метод}
      \item{Интерфейс - напълно абстрактен клас}
    \end{itemize}
  \end{frame}

  \section{Полиморфизъм}

  \begin{frame}
    \frametitle{Полиморфизъм}
    \begin{itemize}
      \item{Dynamic vs static binding}
      \item{Method overloading}
      \item{Method overriding}
    \end{itemize}
  \end{frame}

  \section{Енкапсулация на данни}

  \begin{frame}
    \frametitle{Енкапсулация на данни}
    \framesubtitle{Когато моето си е мое}
    Потребителят на нашият код трябва да знае само какво правим, а не как го правим.
  \end{frame}

  \begin{frame}
    \includegraphics[width=250px]{img/scumbag_teacher}
  \end{frame}

\end{document}

